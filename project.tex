%
% MATH 310 -- Risto Atanasov
% Western Carolina University
%
% Joseph Hunt, Bobby Wertman, 
%
\documentclass{article}
\usepackage{cite}
\usepackage{listings}
\usepackage{color}

\title{Sorting Algorithms}
\author{Joseph Randall Hunt\\
Bobby Wertman\\
Western Carolina University,\\
Cullowhee, North Carolina\\
}
\date{\today}

\begin{document}
\maketitle
\section{Project Proposal}
   \subsection{Goals}
   \subsection{Plan}
\section{Sorts}
   \subsection{Merge Sort}
      \subsubsection{Algorithm}
        Merge sort is a comparison-based sorting algorithm, based on the
        divide-and-conquer design.  Its average and worst cases are both $n
        log(n)$, and its best case is $\Omega(n)$.  Invented in 1945 by John
        von Neumann, it exploits the fact that combining two lists of sorted
        data is a linear-time process.  \cite{introalg}
      \subsubsection{Efficiency}
        Merge sort's complexity is $n log(n)$ for all cases, but the number of
        computations performed changes between best and worst cases.  In the
        all cases, Merge Sort performs $log(n)$ splits of $n$ elements, putting
        the efficiency of this phase at $n log(n)$.  However, when merging, the
        number of operations varies based on the data.
        \paragraph{Worst Case}
          In the worst case, the input data is interleaved in such a way that
          for each step of the merging process, both lists of elements are
          traversed in full before the merge is complete.  This results in $n log(n)$ comparisons, and brings the total number of operations for the sort to $2n log(n)$, so the efficiency class is $O(n log(n))$.
        \paragraph{Best Case}
        \paragraph{Average Case}
        \paragraph{Code}
        The actual code for the algorithm is available in \textbf{Listing
        \ref{code:mergesort}}.  This particular implementation uses an
        optimization that switches to insertion sort on small arrays.  This
        speeds up the algorithm because it allows the small data set to fit
        entirely in cache along with the small amount of code associated with
        the insertion sort algorithm.  See \textbf{Listing
        \ref{code:mergesort-parallel}} for a parallel implementation of merge
        sort.
      \subsubsection{Applications}
        Merge sort is useful in applications where the data set will not fit
        entirely into memory.  This allows for the data to be read in from disk
        and sorted as it is read, thus requiring a very small memory footprint.
        In addition, when time complexity needs to be guaranteed, merge sort is
        preferable over quicksort, as quicksort's worst case is $O(n^2)$.
   \subsection{Quick Sort}
      \subsubsection{Algorithm}
      \subsubsection{Efficiency}
      \subsubsection{Applications}
   \subsection{Shell Sort}
      \subsubsection{Algorithm}
      \subsubsection{Efficiency}
      \subsubsection{Applications}
   \subsection{Comparing and Contrasting}
\section{Conclusions}

\appendix
\section{Code Examples}
\lstset{
    language=Go,
    basicstyle=\footnotesize,
    keywordstyle=\bfseries\color[rgb]{0.8,0.6,0.1},
    commentstyle=\scriptsize\color[rgb]{0.133,0.133,0.545},
    stringstyle=\ttfamily\color[rgb]{0.627,0.126,0.941},
    numbers=left,
    frame=single,
    stepnumber=1,
    identifierstyle=\ttfamily,
    tabsize=2
}
\lstinputlisting[caption={Go implementation of MergeSort},label=code:mergesort]{src/merge.go}
\lstinputlisting[caption={Parallel implementation of MergeSort in Go},label=code:mergesort-parallel]{src/parallel.go}

\bibliography{sources}
\bibliographystyle{plain}
\end{document}
