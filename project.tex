%
% MATH 310 -- Risto Atanasov
% Western Carolina University
%
% Joseph Hunt, Bobby Wertman, Tyler McKinney, 
%
\documentclass{article}
\usepackage{cite}
\usepackage{listings}
\usepackage{color}
\usepackage{appendix}
\usepackage[letterpaper]{geometry}
\usepackage{verbatim}
\title{Sorting Algorithms}
\author{Joseph Randall Hunt\\
Tyler McKinney\\
Bobby Wertman\\
Marco Anton\\
Western Carolina University,\\
Cullowhee, North Carolina\\
}
\date{\today}

\begin{document}
\maketitle
\section{Project Proposal}
   \subsection{Goals}
   \subsection{Plan}
\section{Sorts}
   \subsection{Merge Sort}
      \subsubsection{Algorithm}
        Merge sort is a comparison-based sorting algorithm, based on the
        divide-and-conquer design.  Its average and worst cases are both $n
        log(n)$, and its best case is $\Omega(n)$.  Invented in 1945 by John
        von Neumann, it exploits the fact that combining two lists of sorted
        data is a linear-time process.  \cite{introalg}
      \subsubsection{Efficiency}
        Merge sort's complexity is $n log(n)$ for all cases, but the number of
        computations performed changes between best and worst cases.  In the
        all cases, Merge Sort performs $log(n)$ splits of $n$ elements, putting
        the efficiency of this phase at $n log(n)$.  However, when merging, the
        number of operations varies based on the data.
        \paragraph{Worst Case}
          In the worst case, the input data is interleaved in such a way that
          for each step of the merging process, both lists of elements are
          traversed in full before the merge is complete.  This results in $n$
          comparisons over $log(n)$ levels (complexity $n log(n)$, and brings
          the total number of operations for the sort to $2n log(n)$, so the
          efficiency class is $O(n log(n))$.
        \paragraph{Best Case}
          In the best case, the input data is already in order, allowing the
          merge operations to traverse only the first list, ignoring the second.
          This cuts the number of operations in half compared to the worst
          case, so it performs $\frac{n}{2}$ comparisons on $log(n)$ levels,
          which is $\frac{n log(n)}{2}$ operations for the merge phase and
          $\frac{3n log(n)}{2}$ for the sort, giving us the complexity of
          $\Omega(n log(n))$.  Supposedly, a variation of Merge Sort exists with
          efficiency class $O(n)$ for already sorted data.
        \paragraph{Code}
          The actual code for the algorithm is available in \textbf{Listing
          \ref{code:mergesort}}.  This particular implementation uses an
          optimization that switches to insertion sort on small arrays.  This
          speeds up the algorithm because it allows the small data set to fit
          entirely in cache along with the small amount of code associated with
          the insertion sort algorithm.  See \textbf{Listing
          \ref{code:mergesort-parallel}} for a parallel implementation of merge
          sort.
      \subsubsection{Applications}
        Merge sort is useful in applications where the data set will not fit
        entirely into memory.  This allows for the data to be read in from disk
        and sorted as it is read, thus requiring a very small memory footprint.
        In addition, when time complexity needs to be guaranteed, merge sort is
        preferable over quicksort, as quicksort's worst case is $O(n^2)$.
   \subsection{Quick Sort}
      \subsubsection{Algorithm}Quicksort is a fast algorithm, developed
by Tony Hoare, that has a 
best and average running time of $O(nlog(n))$ and a worst-case  of
$O(n^2)$.\\
Quicksort first divides a large list into two smaller lists with respect
to a ``pivot" (a pivot is a random item selected from the list). The 
first list contains items that are smaller and the second list contains 
items that are larger than the pivot. Finally, apply quicksort to the
smaller and larger lists and return the ordered list. 
According to Weiss\cite{weiss}  the basic algorithm to sort an array $S$ 
consist of the following four steps:
\begin{enumerate}
\item If the number of elements in $S$ is $0$ or $1$, then return.
\item Pick any element $v$ in $S$. This is called the \textbf{pivot}.
\item \textbf{Partition} $S - \{v\}$ (the remaining elements in $S$)
into two disjoint groups: $S_1 = \{x \in S - \{v\}|x\leq v\}$, and $S_2
= \{x \in S - \{v\}|x \geq v\}$.
\item Return $\{$quicksort$(S_1)$ followed by $v$ followed by
quicksort$(S_2)\}$
\end{enumerate}
\subsubsection{Efficiency}
Like mergesort, quicksort is recursive; therefore, its analysis requires
solving a recurrence formula.\\
\begin{itemize}
\item\emph{Worst Case Analysis}\\
To set the recurrence relation on this case, one assumes that the pivot
that is chosen is always the smallest in the list. Therefore, there is 
only going to be one list (the list containing the largest elements).
The recurrence relation on the worst case is:
$$T(1) = 1$$
$$T(n) = T(n-1) + Cn\mbox{, } n > 1$$
Where $T(n-1)$ are the recursions and $c*n$ is the linear time spent in
the partition (selecting a pivot); moreover, in the base case, we just
sort $0$ or $1$ item, we can do this in constant time. To solve this
recursion, 
the method of Backward substitution is used:\\
$$\begin{array}{lclcl}
 &  & &  & substitute: T(n - 1) = T(n - 2) + C(n - 1)\\
T(n - 2) + C(n - 1) + Cn & & &;& substitute: T(n - 2) = T(n - 3) + C(n -
2)\\
T(n - 3)) + C(n - 2) + C(n - 1) + Cn & & &;& substitute: T(n - 3) = T(n
- 4) \\
\vdots & & & & \\
\end{array}$$
$=> T(n - (n - 1)) = C(n - n) + C(n-(n - 1)) + ... + Cn$\\
$= \displaystyle\sum_{i = 2}^{n}Ci = C\left(\frac{n(n + 1)}{2}\right)$
Therefore, in this case quicksort is of $O(n^2)$\\

\item\emph{Best Case Analysis}\\
For the best case analysis we have to choose the pivot such that it is
always in the middle. Therefore our new recurrence relation is:
$$T(n) = 2T(n/2) + Cn$$
$$T(1) = 1$$
Note that the recursion went from $T(n - 1)$ to $T(n/2)$. This is
because
of the selection of the pivot; the pivot has been selected such that the
list is divided into two equal and smaller lists. These smaller lists
are
performed recursively (this is the reason for the $2$ in front of the
recurence relation) until $T(1)$ where we only return the element.\\
This recurrence relation could be solved using the ``Master
Theorem"\cite{levitin}.\\
$$a = 2; b = 2; d = 1$$
$$\mbox{CASE II: } a = b^d  =>  2 = 2^1$$
Therefore the efficiency class in this case is: $\Theta(n^dlogn) =
\Theta(nlogn)$\\

%\item\emph{Average Case Analysis}\\
\end{itemize}
As analysed before, the pivot can be chosen in many ways; however, this 
might make the quicksort algorithm to perform poorly. As stated in the 
\emph{Best Case Analysis}, in order for quicksort to perform better, 
one has to select a pivot to be in the center of the array, but this 
might be an issue when we have a list of odd elements. 
An efficient way of choosing the pivot is choosing the element of 
intermediate value between three elements in the list, more
specifically, 
the three elements chosen from
the list have to be the first, last and middle possition (middle
element would be $\lceil N/2 \rceil$ where $N =$ number of elements in a 
list). After selecting them, they need to be sorted and the pivot
selected 
would be the middle element. This method is called \emph{Median-of-Three
partitioning}. Not only this method will help choose a good pivot, but
also, will help reduce the number of elements to check. The elments in
the
first and last position are in the correct side of their corresponding
lists; therefore, they don't have to be checked when performing
quicksort.\\
\begin{itemize}
\item QuickSort Example.\\
The following list is Sorted using the quicksort algorithm:\\
$$S = 3, 1, 4, 1, 5, 9, 2, 6, 5, 3, 5$$

\begin{itemize}
\item Find "Pivot''\\
Left index = $0$; right index = $10$; center = $5$\\
Sort left, right and center (compare the elements of these indexes and
put the largest on the right index, and the smallest on the left
index).\\
The pivot is the element in position ``center". For the next step, one
needs to put the pivot on position $(n - 1)$ of the array ($n$ is the
size of the array). This yields the following array $3 1 4 1 5 3 2 6
5 4 5 (5) 9$. The item between parentheses is the pivot.\\
\item Partition array into two groups:
Set i to $1$ and j to $n - 2$ (note that left and  right items are on
the
correct side of the array).\\
Transverse through the array until i is greater or equal to the pivot
and j
is smaller or equal to the pivot; in this case until $i = 4$ and
$j = 8$, and swap the items in i and j. This produces the following
array:
$3 1 4 1 5 4 2 6 5 (5) 9$\\
On the next iteration i and j have crossed. No swap is performed;
however, we need to restore the pivot to its correct position (i).\\
Pivot in correct position: $3 1 4 1 5 3 2 (5) 5 6 9$\\
After this step, it is known that the pivot is sorted (smaller elements
are on left of pivot and bigger elements are on right of pivot).\\
\item Quicksort the remaining two groups \\
We now proceed to sort elements to the left of the pivot (smaller
elements) and elements to the right of the pivot (bigger
elements).\\
\end{itemize}

\centering\begin{tabular}{|c|c|c|c|c|c|c|c|c|c|c|}
\hline
$3$ & $1$ & $4$ & $1$ & $5$ & $3$ & $2$ & $(5)$ & $5$ & $6$ & $9$\\
\hline
$3$ & $1$ & $4$ & $(1)$ & $5$ & $3$ & $2$ & $|5|$ & $5$ & $6$ &
$9$\\ \hline
$1$ & $1$ & $4$ & $(3)$ & $5$ & $3$ & $2$ & $|5|$ & $5$ & $6$ &
$9$\\ \hline
$1$ & $1$ & $4$ & $(2)$ & $5$ & $3$ & $3$ & $|5|$ & $5$ & $6$ &
$9$\\ \hline
$1$ & $1$ & $4$ & $3$ & $5$ & $(2)$ & $3$ & $|5|$ & $5$ & $6$ &
$9$\\ \hline
$1$ & $1$ & $|2|$ & $3$ & $5$ & $4$ & $3$ & $|5|$ & $5$ & $6$ &
$9$\\ \hline
$|1|$ & $|1|$ & $|2|$ & $3$ & $5$ & $4$ & $3$ & $|5|$ & $5$ & $6$ &
$9$\\ \hline
$|1|$ & $|1|$ & $|2|$ & $3$ & $(5)$ & $4$ & $3$ & $|5|$ & $5$ & $6$
& $9$\\ \hline
$|1|$ & $|1|$ & $|2|$ & $3$ & $(3)$ & $4$ & $5$ & $|5|$ & $5$ & $6$
& $9$\\ \hline
$|1|$ & $|1|$ & $|2|$ & $3$ & $4$ & $(3)$ & $5$ & $|5|$ & $5$ & $6$
& $9$\\ \hline
$|1|$ & $|1|$ & $|2|$ & $|3|$ & $|3|$ & $|4|$ & $|5|$ & $|5|$ &
$|5|$ &
$|6|$ & $|9|$\\ \hline
\end{tabular}
\end{itemize}
\subsubsection{Applications}
   \subsection{Shell Sort}
      \subsubsection{Algorithm}
      \subsubsection{Efficiency}
      \subsubsection{Applications}
   \subsection{Comparing and Contrasting}
\section{Conclusions}

\begin{appendices}
\section{Code Examples}
\lstset{
    language=Go,
    basicstyle=\footnotesize,
    keywordstyle=\bfseries\color[rgb]{0.8,0.6,0.1},
    commentstyle=\scriptsize\color[rgb]{0.133,0.133,0.545},
    stringstyle=\ttfamily\color[rgb]{0.627,0.126,0.941},
    numbers=left,
    frame=single,
    stepnumber=1,
    identifierstyle=\ttfamily,
    tabsize=2
}
\lstinputlisting[caption={Go implementation of MergeSort},label=code:mergesort]{src/merge.go}
\lstinputlisting[caption={Parallel implementation of MergeSort in Go},label=code:mergesort-parallel]{src/parallel.go}
\verbatiminput{src/Quicksort.java}
\end{appendices}

\bibliography{sources}
\bibliographystyle{plain}
\end{document}
